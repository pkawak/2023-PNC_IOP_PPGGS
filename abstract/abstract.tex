Venue: IOP Polymer Physics Group Graduate Symposium Friday, September 8, 2023 

9:10 AM - 9:20 AM 

Talk: Dissecting the Payne Effect: How Filler-Polymer Competition Reinforces Elastomeric Nanocomposites 

Abstract: For nearly a century, the precise origin and mechanisms of nanoparticulate reinforcement of elastomeric mechanical response have been a major open question in polymer and nanomaterials science. Additionally, a complete understanding of why nanoparticle additives cause strain softening, also known as the Payne effect, in rubber is still lacking. The literature points to several reinforcement hypotheses, most prominent of which is the glassy bridges hypothesis, where frozen polymer around filler particles forms a network between particles. To elucidate mechanisms at high strain, we conduct nonequilibrium molecular dynamics simulations. We focus on extensional deformation of a coarse-grained model of nanoparticle clusters interspersed in a crosslinked polymer matrix at high temperatures, stretching up to 200\%. Nanoparticles, forming an icosahedral shape, are composed of bonded beads within four shells and are grouped in clusters of seven. Relative to a neat elastomer, stress-strain response curves indicate initial large reinforcement in the linear regime, followed by softening. In the linear regime, reinforcement occurs via network formation of filler within the polymer matrix. At larger strains, the filler network yields and stress response transitions from solid‐like nanoparticulate response to plastic nanoparticulate purely-dissipative response. Our experiments further reveal that reinforcement occurs from extension at an unfavored polymer volume due to a Poisson-ratio mismatch and unfavored normal compression of physically contacting filler particles. Importantly, polymers near the surface of particles behave like the bulk at this high temperature, indicating that glassy bridges are not instrumental at these conditions. Despite this, filled elastomers still exhibit enhancement - without glassy bridges or hydrodynamic effects - lending evidence to multi-mechanism reinforcement theories. The observation of a plastic response emerging from the interplay of coexisting hybrid granular/polymeric solids emphasises that a new paradigm is necessary to further advance this field: elastomer nanocomposites exist at the intersection of polymer physics and granular physics. 
